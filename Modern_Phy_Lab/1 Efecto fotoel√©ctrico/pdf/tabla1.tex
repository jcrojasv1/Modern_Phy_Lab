\begin{table}[H]
    \centering
    \resizebox{8cm}{!} {
    \begin{ruledtabular}
    \begin{tabular}{ccccc}
    \hline
        Corriente & Azul  & Verde & Amarillo & Rojo \\ 
        \hline
        $I [ 	\times 10^{-8} A$] & $[V]$ &  $[V]$ &  $[V]$ &  $[V]$ \\ 
        \hline
        10.00  &0.774 & 0.65 & 0.29 & 0.17 \\ 
        9.50 & 0.776 & 0.65 & 0.30 & 0.18 \\ 
        9.00 & 0.778 & 0.65 & 0.31 & 0.19 \\ 
        8.50 & 0.779 & 0.66 & 0.32 & 0.19 \\ 
        8.00 & 0.780 & 0.66 & 0.32 & 0.19 \\ 
        7.50 & 0.781 & 0.66 & 0.33 & 0.20 \\ 
        7.00 & 0.782 & 0.66 & 0.33 & 0.20 \\ 
        6.50 & 0.783 & 0.66 & 0.34 & 0.21 \\ 
        6.00 & 0.784 & 0.67 & 0.35 & 0.22 \\ 
        5.50 & 0.785 & 0.67 & 0.36 & 0.22 \\ 
        5.00 & 0.786 & 0.67 & 0.37 & 0.23 \\ 
        4.50 & 0.787 & 0.67 & 0.37 & 0.23 \\ 
        4.00 & 0.788 & 0.67 & 0.38 & 0.24 \\ 
        3.50 & 0.789 & 0.68 & 0.39 & 0.25 \\ 
        3.00 & 0.789 & 0.68 & 0.40 & 0.25 \\ 
        2.50 & 0.790 & 0.68 & 0.41 & 0.26 \\ 
        2.00 & 0.791 & 0.68 & 0.42 & 0.27 \\ 
        1.50 & 0.792 & 0.68 & 0.44 & 0.28 \\ 
        1.00 & 0.793 & 0.68 & 0.46 & 0.29 \\ 
        0.50 & 0.794 & 0.69 & 0.48 & 0.31 \\ 
        0.00 & 0.795 & 0.69 & 0.51 & 0.33 \\ 
    \end{tabular}
    \end{ruledtabular}
    }
    \caption{Datos medidos en el experimento de efecto fotoeléctrico. En cada columna de color del LED presenta el muestreo del  valor del voltaje de frenado y la corriente que se produce. La incertidumbre de la medida del voltaje $\sigma_{V} = \pm 0.001 V$ y la de la corriente $\sigma_{I} = \pm 0.25\times 10^{-8} A$. }
    \label{table:tabla_datos}
\end{table}